\documentclass[12pt]{article}
\setlength{\oddsidemargin}{0in}
\setlength{\evensidemargin}{0in}
\setlength{\textwidth}{6.5in}
\setlength{\parindent}{0in}
\setlength{\parskip}{\baselineskip}
\setlength{\headsep}{3pt}

\usepackage{amsmath,amsfonts,amssymb}

\title{Homework 1}

\begin{document}

Elena Yang \\
Homework 1 \\
Math 305 \\
September 10, 2020

\hrulefill

\section{Problem 1}
The problem states that we should determine if the following statement is true: \\
A 5 gallon container and a 13 gallon container can be used to accurately measure off 2 gallons of liquid.

The claim is true. The shortest procedure would be to fill the 5 gallon container and pour all the contents into the 13 gallon container, and repeat this action twice so you are left with an empty 5 gallon container and a 13 gallon container with 10 gallons of water. Then fill the 5 gallon container once again, and pour the contents into the 13 gallon container until it is full. Then you would have 2 gallons of water remaining in the 5 gallon container.

We can write an equation to represent the problem:
\begin{equation*}
    13x + 5y = 2
\end{equation*}
If $x$ or $y$ is negative, that means we are emptying a container of $x$ or $y$ gallons respectively. If $x$ or $y$ is positive, that means we fill the container.
The goal is to find some combination of $x$, $y$ such that the equation holds true. 
We can apply the extended Euclidean Algorithm to solve the equation:
\begin{align*}
    \label{Forwards} 
    13 = 5(2) + 3\\
    5 = 3(1) + 1
\end{align*}
Then work through the Algorithm backwards to find coefficient that solve the original problem:
\begin{equation*}
\begin{split}
2 & =  5 - 3(1)\\
 & =  5 - (13 - 5(2))\\
 & = 5(3) + 13(-1)\\
\end{split}
\end{equation*}
We see that the 5 gallon container has a coefficient of 3, so we would fill that container 3 times, and the 13 gallon container has a coefficient of -1, so we would empty that container 1 time.
\end{document}